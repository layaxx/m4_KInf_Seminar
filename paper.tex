% This must be in the first 5 lines to tell arXiv to use pdfLaTeX, which is strongly recommended.
\pdfoutput=1
% In particular, the hyperref package requires pdfLaTeX in order to break URLs across lines.

\documentclass[11pt]{article}

% Remove the "review" option to generate the final version.
\usepackage[review]{acl_mod}

% Standard package includes
\usepackage{times}
\usepackage{latexsym}
\usepackage{graphicx}

% For proper rendering and hyphenation of words containing Latin characters (including in bib files)
\usepackage[T1]{fontenc}
% For Vietnamese characters
% \usepackage[T5]{fontenc}
% See https://www.latex-project.org/help/documentation/encguide.pdf for other character sets

% This assumes your files are encoded as UTF8
\usepackage[utf8]{inputenc}

% This is not strictly necessary, and may be commented out,
% but it will improve the layout of the manuscript,
% and will typically save some space.
\usepackage{microtype}

% This package helps printing tables more nicely.
\usepackage{booktabs}

% Uncomment and modify for custom submission date
% \renewcommand{\subdate}{October 3, 2022}

%--------------------------------------------------------------------------------------------------
%--------------------------------------------------------------------------------------------------
% Modify title, author information, module name here

\def\paperTitle{Semantic Publishing}
\def\authors{Yannick Lang}
\def\matriculationNumber{1995498}
\def\moduleName{KInf-Seminar-M}

%--------------------------------------------------------------------------------------------------
%--------------------------------------------------------------------------------------------------


\title{\paperTitle}

\author{\authors \\
  Matrikelnummer: \matriculationNumber \\
  \moduleName \\
  Fakultät Wirtschaftsinformatik \& Angewandte Informatik \\
  Otto-Friedrich-Universität Bamberg}

\begin{document}
\maketitle
\begin{abstract}
    Lorem Ipsum
\end{abstract}


\renewcommand*\contentsname{Inhaltsverzeichnis}
{\footnotesize
  \tableofcontents}

\section{Einführung}
\label{sec:einfuehrung}

Besonders in sehr dynamischen Forschungsbereichen, wie beispielsweise dem Machine Learning, ist es wichtig, stets auf dem aktuellen Stand der Forschung zu bleiben.
Dies ist jedoch aufgrund der hohen Anzahl an wissenschaftlichen Publikationen, die jährlich veröffentlicht werden, eine Herausforderung; herkömmliche Literaturübersichten können bereits nach wenigen Monaten überholt sein.
Eine Möglichkeit, um den Überblick zu behalten, sind sogenannte \textit{Living Literature Reviews} (LLRs).
Diese werden regelmäßig aktualisiert und enthalten eine Zusammenfassung der aktuellen Forschungsergebnisse zu einem bestimmten Thema.

In dieser Arbeit wird zuerst der Anwendungsfall näher erläutert.
Anschließend werden zwei Ansätze zur Erstellung und Aktualisierung von LLRs vorgestellt und miteinander verglichen:
\textit{Genuine Semantic Publishing} (GSP, \cite{kuhn2017genuine}), beruhend auf von den Autoren selbst erstellten und veröffentlichten, maschinenverarbeitbaren Informationen über Ihre Forschungsbeiträge und SCICERO, eine auf Natural Language Processing und Transformer-Modellen beruhender Ansatz.
Dabei wird untersucht, wie gut die beiden Ansätze in der Lage sind, den aktuellen Stand der Forschung abzubilden und welche Vor- und Nachteile sie haben.

\section{Living Literature Reviews}
\label{sec:living-literature-reviews}


Literaturübersichten sind ein wichtiges Instrument, um den Überblick über den aktuellen Stand der Forschung zu behalten.
Traditionell werden diese in Form von wissenschaftlichen Publikationen veröffentlicht, die eine Zusammenfassung der relevanten Arbeiten zu einem bestimmten Thema enthalten.
Zusätzlich können auch Empfehlungen oder Metaanalysen ein Bestandteil sein, um die Ergebnisse der verschiedenen Studien zu vergleichen und zu bewerten \cite{elliott2017living}.

\paragraph{Living Literature Reviews}

Besonders in sehr dynamischen Forschungsbereichen, wie beispielsweise dem Machine Learning, können  Literaturübersichten zwar besonders hilfreich sein, aber auch schnell veralten.
Eine Möglichkeit, um den Überblick zu behalten, sind sogenannte \textit{Living Literature Reviews} (LLRs, auch: Living Systematic Reviews), also Literaturübersichten, die regelmäßig an den aktuellen Forschungsstand angepasst werden \cite{living-lit-review}.
Wann solche Aktualisierungen stattfinden, kann dabei variieren, etwa in regelmäßigen Abständen oder bei wichtigen neuen Entwicklungen \cite{cochrane}.

Um den Aufwand für die Aktualisierung zu minimieren, gibt es Bestrebungen, dafür auf maschinelle Unterstützung zurückzugreifen.
So können beispielsweise Nanopublikation \cite{nanopubs} verwendet werden, wodurch zu einer Literaturübersicht leicht neue Publikationen hinzugefügt werden können, ohne die gesamte Übersicht neu verfassen zu müssen \cite{living-lit-review}.
Dadurch werden auch innovativere Formen von Literaturübersichten möglich, die beispielsweise interaktiv sind und so den Nutzern erlauben, die Übersicht nach ihren eigenen Bedürfnissen zu filtern oder Veränderungen im Zeitverlauf zu betrachten \cite{living-lit-review}.
Damit sind solche Reviews auch nicht an traditionelle Publikationsformen gebunden, sondern können neben Webseiten oder interaktiven Grafiken beispielsweise auch selbst in maschinenlesbaren Datenformaten veröffentlicht werden \cite{linked-lit-review}.


\paragraph{Wissensgraphen in Living Literature Reviews}

Wissensgraphen könnten an mehreren Stellen in den Prozess der Erstellung von Living Literature Reviews eingebunden werden.
So könnten sie beispielsweise dazu genutzt werden, Literatur ausfindig zu machen, die für die Erstellung eines Reviews relevant ist.
Dabei könnten Wissensgraphen helfen, indem sie die Beziehungen zwischen verschiedenen Publikationen und Forschungsfeldern abbilden und so die Suche nach verwandten Arbeiten erleichtern.
Einerseits könnte das über die Verwendung der Zitationen und Referenzen geschehen.
Durch einen Fokus auf die in den Publikationen verwendeten Begriffe und Konzepte könnten Wissensgraphen aber auch dabei helfen, Publikationen zu finden, die sich mit ähnlichen Themen beschäftigen, aber nicht gegenseitig zitieren, etwa, weil den Autoren selbst die andere Arbeit nicht bekannt ist \cite{citation-recommendation}.
Wissensgraphen können darüber hinaus auch zur Thesengenerierung und -validierung genutzt werden, um so die Ergebnisse der Literaturrecherche zu strukturieren und gegebenenfalls anzupassen, wenn neue relevante Inhalte hinzugefügt werden \cite{DESSI2022109945}.
Schließlich können die in Wissensgraphen oft enthaltenen Herkunftsinformationen verwendet werden, um die Qualität der gefundenen Publikationen zu bewerten, indem beispielsweise die Reputation der Autoren oder der Publikationsorte berücksichtigt wird.

Als Kernfrage verbleibt, woher die zugrundeliegenden maschinenverarbeitbaren Daten kommen.
In den folgenden Kapiteln werden hierfür zwei potentielle Lösungsansätze vorgestellt.
\section{Genuine Semantic Publishing}
\label{sec:genuine-semantic-publishing}

Im folgenden wird das von \citet{kuhn2017genuine} vorgestellte Konzept des Genuine Semantic Publishing erläutert und dessen potentielle Einsatzgebiete für Living Literature Reviews diskutiert.

\subsection{Motivation}
\label{subsec:motivation}

Die Autoren argumentieren, dass semantic publishing in früheren Arbeiten oft lediglich als Anreicherung von bestehenden Publikationen mit (Meta-) Daten betrachtet wurde.
Die gängige Definition von Semantic Publishing als \textit{alles, was die Bedeutung veröffentlichter Artikel aufwertet, automatisches Auffinden von Artikeln oder Verbinden von Artikeln mit verwandtem Inhalt ermöglicht, Zugang zu Daten in verwertbarer Form verschafft oder Wiederverwendung von Daten zwischen Artikeln erleichtert} (im Original: "anything that enhances the meaning of a published journal article, facilitates its automated discovery, enables its linking to semantically related articles, provides access to data within the article in actionable form, or facilitates integration of data between papers") \cite{kuhn2017genuine} weißen die Autoren jedoch als gleichzeitig über- und unterspezifisch zurück.
Ihrer Einschätzung nach würde das Hinzufügen von Schlagwörtern bereits ausreichen, um unter die Definition zu fallen (ermöglicht das Verbinden inhaltlich verwandter Artikel).
Andererseits könnten Publikationen, die zwar Forschungsergebnisse als semantischen Repräsentationen abbilden, aber auf den Fließtext verzichten, dieser Definition nach nicht als Semantische Publikationen bezeichnet werden.

Um diese Schwächen zu adressieren, schlagen die Autoren vor, den gesamten Publikationsprozess zu überdenken.
Um zu verdeutlichen, das das damit Beschriebene näher an der ursprünglichen Grundidee von Semantic Publishing ist, nennen Sie dieses Konzept Genuine Semantic Publishing.
Als semantisch verstehen die Autoren dabei \textit{eine formale logikbasierte Darstellung der Bedeutung des Inhalts} (im Original: "carrying a formal logic-based representation of the content's meaning").

\subsection{Definition}
\label{subsec:definition}

Um die Abgrenzung zu anderen Ansätzen zu realisieren, definieren die Autoren die folgenden fünf Prinzipien, die eine Publikation erfüllen muss, um als \textit{genuine semantic} zu gelten: (1) Machine interpretable, (2) essential coverage, (3) authenticity, (4) primary component, und (5) fine-grained \& light-weight.

\paragraph{Machine interpretable}
\label{par:machine-interpretable}
Die Interpretation, nicht nur das Lesen, des Inhalts der Publikation muss durch eine Maschine möglich sein.

\paragraph{Essential coverage}
\label{par:essential-coverage}
Mindestens die Kerninformationen der Publikation müssen in semantischer Repräsentation vorliegen.
Zusammen mit dem Prinzip der \textit{Machine interpretability} erlaubt dies eine automatisierte Verarbeitung der Inhalte, zum Beispiel ein Zusammenfassen und Schlussfolgern auf der Grundlage wissenschaftlicher Erkenntnisse \cite{kuhn2017genuine}.
Das dieser Aspekt von besonderer Bedeutung für den Einsatz in Living Literature Reviews ist, wird in Abschnitt \ref{subsec:zusammenhang-mit-living-literature-reviews} diskutiert.

\paragraph{Authenticity}
\label{par:authenticity}
Die semantische Repräsentation müssen von Experten \textemdash im Regelfall den Autoren selbst \textemdash erstellt und veröffentlicht werden.

\paragraph{Primary component}
\label{par:primary-component}
Die semantischen Inhalte dürfen weder nachträglich veröffentlicht werden, noch sollen Sie ein bloßer Anhang oder von anderen Teilen der Publikation abhängig sein.

\paragraph{Fine-grained \& light-weight}
\label{par:fine-grained-light-weight}
Einzelne Inhalte sollen leicht ergänzt oder korrigiert werden können.

\subsection{Zusammenhang mit Living Literature Reviews}
\label{subsec:zusammenhang-mit-living-literature-reviews}

Die Autoren betrachten Genuine Semantic Publishing als Möglichkeit, "komplexe Fragen zu beantworten oder interaktive Karten der wissenschaftlichen Literatur zu erstellen" (im Original: "to answer complex questions or produce interactive science maps") \cite{kuhn2017genuine}.
Diese Vision ist eng mit dem Konzept der Living Literature Reviews verbunden, die regelmäßig aktualisiert werden und eine Zusammenfassung des aktuellen Forschungsstandes zu einem bestimmten Thema bieten.
So erkennen die Autoren selbst das Potential, einen schnellen und einfachen aber trotzdem umfassenden und zutreffenden Überblick über den aktuellen Stand der Forschung zu ermöglichen.

\section{SCICERO}
\label{sec:scicero}

Ähnlich wie \citet{kuhn2017genuine} verfolgen auch \citet{DESSI2022109945} das Ziel, die Sondierung, Verbindung und Analyse wissenschaftlicher Publikationen zu erleichtern.
Auch Sie sind der Ansicht, dass angesichts der Menge an wissenschaftlichen Publikationen, die jährlich veröffentlicht werden, maschinelle Unterstützung notwendig ist, um diese Ziele zu erreichen.



\subsection{Motivation}

Im Gegensatz zu Genuine Semantic Publishing, bei dem ganz Explizit die Autoren selbst maschinenverarbeitbare Informationen über ihre Forschungsbeiträge erstellen und veröffentlichen müssen, ist SCICERO ein extraktiver Ansatz, beruhend auf auf Natural Language Processing und Transformer-Modellen.

Gegenüber von Menschen annotierten Verfahren bietet das den Vorteil, schnell auf große Mengen an Texten angewendet werden zu können.
Der Vorteil gegenüber anderen KI-basierten Ansätzen ist, dass SCICERO nicht nur für die Analyse einzelner Publikationen, sondern explizit für den Vergleich und die Verknüpfung von Publikationen entwickelt wurde.
Hierbei ergeben sich für maschinelle Verfahren zwei Herausforderungen: (1) die Identifikation von relevanten Entitäten und deren Relationen innerhalb eines Textes und (2) die Verknüpfung dieser Informationen zwischen Texten.
Insbesondere bei letzterem ist es wichtig, dass Entitäten zusammengefasst werden, die in verschiedenen Texten unterschiedlich benannt werden, aber dasselbe Konzept beschreiben.

\subsection{Vorgehen}
\label{subsec:scicero-vorgehen}

SCICERO besteht aus drei Schritten: (1) Extraktion, (2) Entitäts- und Relationsbearbeitung und (3) Validierung.

\paragraph{Extraktion}

Im ersten Schritt werden mehrere verschiedene Methoden zur Entitäts- bzw. Relationsextraktion angewendet.
Neben klassischen Methoden des Natural Language Processing, wie zum Beispiel Part-Of-Speech Tagging, kommen hier auch auf Transformer-Modellen basierende Verfahren wie DyGIEpp \cite{wadden-etal-2019-entity} zum Einsatz.
Von welcher Methode eine Relation extrahiert wurde, wird als Teil der Herkunftsinformationen gespeichert.

\paragraph{Entitäts- und Relationsbearbeitung}

Anschließend werden Entitäten dedupliziert und Relationen auf eine Ontologie zurückgeführt.
Um die Menge der gefundenen Entitäten zu reduzieren, kommen verschiedene Methoden zum Einsatz.
Unter anderem werden Entitäten mit wenig Informationsgehalt entfernt und Abkürzungen auf ihre Langform zurückgeführt.
Außerdem kommen aus dem Information Retrieval bekannte Verfahren wie das Entfernen sogenannter stop-words und die Lemmatisierung, also die Bildung von Grundformen, zum Einsatz \cite{Ceri2013}.
Anschließend werden die Entitäten auf kanonische Formen zurückgeführt.
Neben Transformern werden hier auch externe Quellen wie DBpedia und Wikidata verwendet, die unter anderem Informationen über Synonyme und alternative Bezeichnungen enthalten.

Auch werden die gefundenen Relationen normalisiert und inhaltlich ähnliche Beziehungen zusammengefasst.

\paragraph{Validierung}

Im letzten Schritt werden die extrahierten Relationen validiert.
Hier soll einerseits sichergestellt werden, dass sie konsistent mit anderen Relationen sind, die das System als korrekt annimmt, und andererseits, dass die Entitäten einer Relation dem \textemdash gemäß der verwendeten Ontologie \textemdash erwarteten Typ entsprechen (z.B. Material, Aufgabe oder Metrik).
Für die erste Validierungsform wird die Tatsache ausgenutzt, dass die Wahrscheinlichkeit, dass eine Relation korrekt ist, mit der Anzahl der Artikel korreliert, aus denen die Relation extrahiert wurde.
Relationen mit hohem Support, die also aus vielen Artikeln stammen, werden als korrekt angenommen und zum fine-tuning eines Transformer-Modells verwendet.
Für Relationen mit wenig Support entscheidet dieses Modell dann, ob sie konsistent mit den als korrekt angenommenen Relationen sind.
Schlussendlich erfolgt eine regelbasierte Validierung, bei der die Entitäten einer Relation auf ihre Typen überprüft werden, beispielsweise kann eine Methode ein Material benutzen (\verb|<methodX usesMaterial materialY>|), nicht aber umgekehrt.
Die verfügbaren Typen sind dabei in einer Ontologie festgelegt.

\subsection{Zusammenhang mit Living Literature Reviews}

Die Autoren haben SCICERO explizit für die Generierung von Wissensgraphen zu wissenschaftlichen Forschungsfeldern entwickelt.
So haben Sie aus 6.7 Millionen wissenschaftlichen Publikationen aus dem Bereich der Informatik den Wissensgraphen \textit{CS-KG} \cite{cskg} erstellt.
Mit einem ähnlichen Ansatz hatten die Autoren zuvor bereits einen Wissensgraphen zu dem Subfeld der Künstlichen Intelligenz erstellt \cite{aikg} und damit die Anwendbarkeit auch für kleinere Felder demonstriert.

Je nach intendiertem Forschungsfeld kann die verwendete Ziel-Ontologie angepasst werden.
Statt der auf Informatik spezialisierten Computer Science Knowledge Graph Ontology können beliebige Alternativen \textemdash beispielsweise \textit{Gene Ontology} oder \textit{Mathematics Subject Classification} \cite{DESSI2022109945} \textemdash verwendet werden.
Entsprechend erfordert dies dann auch Anpassungen an den Extraktionsmodulen und Validierungsregeln.


\section{Vergleich}
\label{sec:vergleich}

\section{Fazit}
\label{sec:fazit}

Aus technischer Sicht spricht nichts gegen die Vision des Genuine Semantic Publishings.
Dennoch müsste der gesamte Publikationsprozess verändert werden, traditionelle, Textbasierte Prozesse müssten abgelöst werden durch solche, die den Fokus auf die Erkenntnisse und deren präzise Repräsentation legen.
Da ein solcher Wandel, der die Mitarbeit aller Forschender bedarf, vorerst unrealistisch erscheint, sind die Vorteile des Verfahrens in der praxis irrelevant.

Für den beschriebenen Anwendungsfall bleibt Genuine Semantic Publishing also prinzipiell geeignet, aber eher als eine Vision der Zukunft.
Um sofort, oder auch nur in naher Zukunft, etwas derartiges zu erreichen, führt wohl kein Weg an KI-basierten Ansätzen wie SCICERO vorbei, die auch mit den großen, bereits bestehenden Datenmengen umgehen können, die nicht extra mit dem Ziel erstellt wurden, für Maschinen oder Programme verarbeitbar zu sein.

% Ausblick: Human in the Loop als Kompromiss?




\bibliography{custom}
\bibliographystyle{acl_natbib}

%\appendix
%
%\section{Example Appendix}
%\label{sec:appendix}
%
%This is an appendix.

\end{document}
