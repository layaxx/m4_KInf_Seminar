\begin{abstract}
    Genuine semantic publishing ist ein 2017 von Tobias Kuhn und Michel Dumontier vorgestelltes Konzept, das darauf abzielt, wissenschaftliche Arbeiten maschinenverarbeitbar zu machen.
    Mit Hinblick auf die stetig steigende Menge an wissenschaftlichen Publikationen und der damit einhergehenden Schwierigkeit, diese zu verarbeiten, stellt sich die Frage, ob die Erstellung und Aktualisierung von Literaturübersichten durch den Einsatz von Genuine Semantic Publishing erleichtert werden kann.
    In dieser Arbeit wird die Eignung von Genuine Semantic Publishing für Knowledge-Graph-basierte Living Literature Reviews untersucht und mit alternativen, KI-basierten Ansätze wie SCICERO verglichen.
\end{abstract}
