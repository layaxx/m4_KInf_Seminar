\section{Living Literature Reviews}
\label{sec:living-literature-reviews}


Literaturübersichten sind ein wichtiges Instrument, um den Überblick über den aktuellen Stand der Forschung zu behalten.
Traditionell werden diese in Form von wissenschaftlichen Publikationen veröffentlicht, die eine Zusammenfassung der relevanten Arbeiten zu einem bestimmten Thema enthalten.
Zusätzlich können auch Empfehlungen oder Metaanalysen ein Bestandteil sein, um die Ergebnisse der verschiedenen Studien zu vergleichen und zu bewerten \cite{elliott2017living}.

\paragraph{Living Literature Reviews}

Besonders in sehr dynamischen Forschungsbereichen, wie beispielsweise dem Machine Learning, können  Literaturübersichten zwar besonders hilfreich sein, aber auch schnell veralten.
Eine Möglichkeit, um den Überblick zu behalten, sind sogenannte \textit{Living Literature Reviews} (LLRs, auch: Living Systematic Reviews), also Literaturübersichten, die regelmäßig an den aktuellen Forschungsstand angepasst werden \cite{living-lit-review}.
Wann solche Aktualisierungen stattfinden, kann dabei variieren, etwa in regelmäßigen Abständen oder bei wichtigen neuen Entwicklungen \cite{cochrane}.

Das Anlegen und Aktualisieren von Literaturübersichten ist jedoch aufwändig und erfordert eine Menge manueller Arbeit.
Neben dem Auffinden relevanter Publikationen müssen diese gelesen und bewertet werden, um zu entscheiden, ob sie in die Übersicht aufgenommen werden sollen.
Anschließend müssen die Ergebnisse der verschiedenen Studien zusammengefasst und bewertet werden, um eine kohärente Übersicht zu erstellen.
Erst dann kann die Übersicht geschrieben und veröffentlicht werden \cite{Brack2022}.


Weil die Menge an wissenschaftlichen Publikationen, die jährlich veröffentlicht werden, stetig steigt, ist es schwierig, diesen Prozess manuell zu bewältigen \cite{knowledge-extraction}.
Andererseits ist auch eine maschinelle Unterstützung mit Hindernissen verbunden, da Publikationen traditionell in natürlicher Sprache verfasst sind und in der Regel als PDF veröffentlicht werden und daher nicht ohne weiteres maschinell verarbeitet werden können \cite{Brack2022}.

Um den Aufwand \textemdash gerade für die Aktualisierung \textemdash zu verringern, gibt es Bestrebungen, maschinelle Unterstützung bei der Arbeit mit großen Mengen wissenschaftlicher Publikationen zu ermöglichen.
So können beispielsweise Nanopublikation \cite{nanopubs} verwendet werden, wodurch zu einer Literaturübersicht leicht neue Publikationen hinzugefügt werden können, ohne die gesamte Übersicht neu verfassen zu müssen \cite{living-lit-review}.
Dadurch werden auch innovativere Formen von Literaturübersichten möglich, die beispielsweise interaktiv sind und so den Nutzern erlauben, die Übersicht nach ihren eigenen Bedürfnissen zu filtern oder Veränderungen im Zeitverlauf zu betrachten \cite{living-lit-review}.
Damit sind solche Reviews auch nicht an traditionelle Publikationsformen gebunden, sondern können neben Webseiten oder interaktiven Grafiken beispielsweise auch selbst in maschinenlesbaren Datenformaten veröffentlicht werden \cite{linked-lit-review}.


\paragraph{Wissensgraphen in Living Literature Reviews}

Wissensgraphen könnten an mehreren Stellen in den Prozess der Erstellung von Living Literature Reviews eingebunden werden.
So könnten sie beispielsweise dazu genutzt werden, Literatur ausfindig zu machen, die für die Erstellung eines Reviews relevant ist.
Dabei könnten Wissensgraphen helfen, indem sie die Beziehungen zwischen verschiedenen Publikationen und Forschungsfeldern abbilden und so die Suche nach verwandten Arbeiten erleichtern.
Einerseits könnte das über die Verwendung der Zitationen und Referenzen geschehen.
Durch einen Fokus auf die in den Publikationen verwendeten Begriffe und Konzepte könnten Wissensgraphen aber auch dabei helfen, Publikationen zu finden, die sich mit ähnlichen Themen beschäftigen, aber nicht gegenseitig zitieren, etwa, weil den Autoren selbst die andere Arbeit nicht bekannt ist \cite{citation-recommendation}.
Wissensgraphen können darüber hinaus auch zur Thesengenerierung und -validierung genutzt werden, um so die Ergebnisse der Literaturrecherche zu strukturieren und gegebenenfalls anzupassen, wenn neue relevante Inhalte hinzugefügt werden \cite{DESSI2022109945}.
Schließlich können die in Wissensgraphen oft enthaltenen Herkunftsinformationen verwendet werden, um die Qualität der gefundenen Publikationen zu bewerten, indem beispielsweise die Reputation der Autoren oder der Publikationsorte berücksichtigt wird.

Als Kernfrage verbleibt, woher die zugrundeliegenden maschinenverarbeitbaren Daten kommen.
In den folgenden Kapiteln werden hierfür zwei potentielle Lösungsansätze vorgestellt.