\section{Vergleich}
\label{sec:vergleich}


Beide Ansätze enthalten Herkunftsinformationen (im Original: \textit{provenance}) über die Forschungsergebnisse.
Bei Genuine Semantic Publishing beziehen sich diese Informationen auf die Autoren selbst, die die semantischen Daten erstellt haben.
Bei SCICERO sind neben den Artikeln, aus denen die Informationen extrahiert wurden, auch die Modelle und Methoden, die zur Extraktion verwendet wurden, Teil der Herkunftsinformationen.

% Vorteil GSP: weniger Rechenleistung notwendig