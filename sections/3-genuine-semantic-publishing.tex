\section{Genuine Semantic Publishing}
\label{sec:genuine-semantic-publishing}

Im folgenden wird das von \citet{kuhn2017genuine} vorgestellte Konzept des Genuine Semantic Publishing erläutert und dessen potentielle Einsatzgebiete für Living Literature Reviews diskutiert.

\subsection{Motivation}
\label{subsec:motivation}

Die Autoren argumentieren, dass semantic publishing in früheren Arbeiten oft lediglich als Anreicherung von bestehenden Publikationen mit (Meta-) Daten betrachtet wurde.
Die gängige Definition von Semantic Publishing als \textit{alles, was die Bedeutung veröffentlichter Artikel aufwertet, automatisches Auffinden von Artikeln oder Verbinden von Artikeln mit verwandtem Inhalt ermöglicht, Zugang zu Daten in verwertbarer Form verschafft oder Wiederverwendung von Daten zwischen Artikeln erleichtert} (im Original: "anything that enhances the meaning of a published journal article, facilitates its automated discovery, enables its linking to semantically related articles, provides access to data within the article in actionable form, or facilitates integration of data between papers") \cite{kuhn2017genuine} weißen die Autoren jedoch als gleichzeitig über- und unterspezifisch zurück.
Ihrer Einschätzung nach würde das Hinzufügen von Schlagwörtern bereits ausreichen, um unter die Definition zu fallen (ermöglicht das Verbinden inhaltlich verwandter Artikel).
Andererseits könnten Publikationen, die zwar Forschungsergebnisse als semantischen Repräsentationen abbilden, aber auf den Fließtext verzichten, dieser Definition nach nicht als Semantische Publikationen bezeichnet werden.

Um diese Schwächen zu adressieren, schlagen die Autoren vor, den gesamten Publikationsprozess zu überdenken.
Um zu verdeutlichen, das das damit Beschriebene näher an der ursprünglichen Grundidee von Semantic Publishing ist, nennen Sie dieses Konzept Genuine Semantic Publishing.
Als semantisch verstehen die Autoren dabei \textit{eine formale logikbasierte Darstellung der Bedeutung des Inhalts} (im Original: "carrying a formal logic-based representation of the content's meaning").

\subsection{Definition}
\label{subsec:definition}

Um die Abgrenzung zu anderen Ansätzen zu realisieren, definieren die Autoren die folgenden fünf Prinzipien, die eine Publikation erfüllen muss, um als \textit{genuine semantic} zu gelten: (1) Machine interpretable, (2) essential coverage, (3) authenticity, (4) primary component, und (5) fine-grained \& light-weight.

\paragraph{Machine interpretable}
\label{par:machine-interpretable}
Die Interpretation, nicht nur das Lesen, des Inhalts der Publikation muss durch eine Maschine möglich sein.

\paragraph{Essential coverage}
\label{par:essential-coverage}
Mindestens die Kerninformationen der Publikation müssen in semantischer Repräsentation vorliegen.
Zusammen mit dem Prinzip der \textit{Machine interpretability} erlaubt dies eine automatisierte Verarbeitung der Inhalte, zum Beispiel ein Zusammenfassen und Schlussfolgern auf der Grundlage wissenschaftlicher Erkenntnisse \cite{kuhn2017genuine}.
Das dieser Aspekt von besonderer Bedeutung für den Einsatz in Living Literature Reviews ist, wird in Abschnitt \ref{subsec:zusammenhang-mit-living-literature-reviews} diskutiert.

\paragraph{Authenticity}
\label{par:authenticity}
Die semantische Repräsentation müssen von Experten \textemdash im Regelfall den Autoren selbst \textemdash erstellt und veröffentlicht werden.

\paragraph{Primary component}
\label{par:primary-component}
Die semantischen Inhalte dürfen weder nachträglich veröffentlicht werden, noch sollen Sie ein bloßer Anhang oder von anderen Teilen der Publikation abhängig sein.

\paragraph{Fine-grained \& light-weight}
\label{par:fine-grained-light-weight}
Einzelne Inhalte sollen leicht ergänzt oder korrigiert werden können.

\subsection{Zusammenhang mit Living Literature Reviews}
\label{subsec:zusammenhang-mit-living-literature-reviews}

Die Autoren betrachten Genuine Semantic Publishing als Möglichkeit, "komplexe Fragen zu beantworten oder interaktive Karten der wissenschaftlichen Literatur zu erstellen" (im Original: "to answer complex questions or produce interactive science maps") \cite{kuhn2017genuine}.
Diese Vision ist eng mit dem Konzept der Living Literature Reviews verbunden, die regelmäßig aktualisiert werden und eine Zusammenfassung des aktuellen Forschungsstandes zu einem bestimmten Thema bieten.
So erkennen die Autoren selbst das Potential, einen schnellen und einfachen aber trotzdem umfassenden und zutreffenden Überblick über den aktuellen Stand der Forschung zu ermöglichen.


% Vergleich von Provenance bei GSP und SCICERO