\section{Fazit}
\label{sec:fazit}

Aus technischer Sicht spricht nichts gegen die Vision des Genuine Semantic Publishings.
Dennoch müsste der gesamte Publikationsprozess verändert werden, traditionelle, Textbasierte Prozesse müssten abgelöst werden durch solche, die den Fokus auf die Erkenntnisse und deren präzise Repräsentation legen.
Da ein solcher Wandel, der die Mitarbeit aller Forschender bedarf, vorerst unrealistisch erscheint, sind die Vorteile des Verfahrens in der praxis irrelevant.

Für den beschriebenen Anwendungsfall bleibt Genuine Semantic Publishing also prinzipiell geeignet, aber eher als eine Vision der Zukunft.
Um sofort, oder auch nur in naher Zukunft, etwas derartiges zu erreichen, führt wohl kein Weg an KI-basierten Ansätzen wie SCICERO vorbei, die auch mit den großen, bereits bestehenden Datenmengen umgehen können, die nicht extra mit dem Ziel erstellt wurden, für Maschinen oder Programme verarbeitbar zu sein.

Insbesondere Angesichts der Tatsache, dass bei KI-basierten Ansätzen wie SCICERO immer die Möglichkeit besteht, dass Fehler gemacht werden, ist es wichtig, dass die Ergebnisse dieser Ansätze regelmäßig überprüft und gegebenenfalls korrigiert werden.
Hier könnten Human-In-The-Loop Ansätze eine zentrale Rolle spielen und so die \textemdash insbesondere die Effizienz betreffenden \textemdash Vorteile von KI-basierten Ansätzen mit der Genauigkeit von manuellen Ansätzen verbinden.
In solchen Ansätzen könnten menschliche Experten die Ergebnisse der KI-Systeme überprüfen und gegebenenfalls korrigieren, um so die Qualität der Ergebnisse zu sichern.

\citet{Tsaneva2024EnhancingSK} zeigen, dass Human-in-the-loop Ansätze auch in der Praxis funktionieren können, allerdings mit Effizienzeinbußen gegenüber dem Vollautomatisierten System.
Zusätzlich zu den bereits genannten Vorteilen, wie der höheren Qualität der Ergebnisse, könnten solche Ansätze auch dazu beitragen, die Akzeptanz von KI-Systemen zu erhöhen, indem sie die Kontrolle über die Ergebnisse wieder in die Hände der Menschen legen.
Andererseits zeigen Sie auch, dass selbst mit Large Language Models (LLMs) als Experten (LLMs-In-the-Loop) die Qualität der Ergebnisse gegenüber dem Ausgangswert gesteigert werden kann, auch wenn die Präzision dieses Ansatzes (85\%) nicht an die der menschlichen Experten (93\%) heranreicht \cite{Tsaneva2024EnhancingSK}.
Die besten Ergebnisse können demnach mit einer Kombination aus LLMs-In-the-Loop und menschlichen Experten erreicht werden, bei dem die menschlichen Experten nur diejenigen Fakten prüfen, bei denen die automatisierten Validierungsmechanismen \textemdash einerseits die Transformer wie in Kapitel \ref{subsec:cicero-vrogehen} beschrieben und andererseits die LLMs \textemdash zu unterschiedlichen Ergebnissen kommen.
Dadurch kann mit minimalem menschlichen Aufwand eine hohe Qualität der Ergebnisse erreicht werden.