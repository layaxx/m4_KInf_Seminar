\section{Fazit}
\label{sec:fazit}

Aus technischer Sicht spricht nichts gegen die Vision des Genuine Semantic Publishings.
Dennoch müsste der gesamte Publikationsprozess verändert werden, traditionelle, Textbasierte Prozesse müssten abgelöst werden durch solche, die den Fokus auf die Erkenntnisse und deren präzise Repräsentation legen.
Da ein solcher Wandel, der die Mitarbeit aller Forschender bedarf, vorerst unrealistisch erscheint, sind die Vorteile des Verfahrens in der praxis irrelevant.

Für den beschriebenen Anwendungsfall bleibt Genuine Semantic Publishing also prinzipiell geeignet, aber eher als eine Vision der Zukunft.
Um sofort, oder auch nur in naher Zukunft, etwas derartiges zu erreichen, führt wohl kein Weg an KI-basierten Ansätzen wie SCICERO vorbei, die auch mit den großen, bereits bestehenden Datenmengen umgehen können, die nicht extra mit dem Ziel erstellt wurden, für Maschinen oder Programme verarbeitbar zu sein.
