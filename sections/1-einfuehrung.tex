\section{Einführung}
\label{sec:einfuehrung}

Besonders in sehr dynamischen Forschungsbereichen, wie beispielsweise dem Machine Learning, ist es wichtig, stets auf dem aktuellen Stand der Forschung zu bleiben.
Dies ist jedoch aufgrund der hohen Anzahl an wissenschaftlichen Publikationen, die jährlich veröffentlicht werden, eine Herausforderung; herkömmliche Literaturübersichten können bereits nach wenigen Monaten überholt sein.
Eine Möglichkeit, um den Überblick zu behalten, sind sogenannte \textit{Living Literature Reviews} (LLRs).
Diese werden regelmäßig aktualisiert und enthalten eine Zusammenfassung der aktuellen Forschungsergebnisse zu einem bestimmten Thema.

In dieser Arbeit werden zwei Ansätze zur Erstellung und Aktualisierung von LLRs vorgestellt und miteinander verglichen:
\textit{Genuine Semantic Publishing} (GSP, \cite{kuhn2017genuine}), beruhend auf von den Autoren selbst erstellten und veröffentlichten, maschinenverarbeitbaren Informationen über Ihre Forschungsbeiträge und SCICERO, eine auf Natural Language Processing und Transformer-Modellen beruhender Ansatz.
Dabei wird untersucht, wie gut die beiden Ansätze in der Lage sind, den aktuellen Stand der Forschung abzubilden und welche Vor- und Nachteile sie haben.
